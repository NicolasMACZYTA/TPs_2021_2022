\documentclass[12pt,a4paper]{article}
\usepackage[utf8]{inputenc}    % Ces deux packages sont utilisés pour l'encodage des caractères
\usepackage[T1]{fontenc}       % 
\usepackage[french]{babel}     % Ce package permet d'utiliser les environnements en Français

\usepackage{tikz}
\usetikzlibrary{automata, positioning, arrows}

\tikzset{
  ->, % makes the edges directed
  >=stealth', % makes the arrow heads bold
  node distance=3cm, % specifies the minimum distance between two nodes. Change if necessary.
  every state/.style={thick, fill=gray!10}, % sets the properties for each 'state' node
  initial text=$ $, % sets the text that appears on the start arrow
}

%**************************************************************************************************
% *
% * Configuration des marges
% *
%**************************************************************************************************

\usepackage{calc}
\newcommand{\marges}[2]{
  \newlength{\HEADHEIGHT}
  \setlength{\HEADHEIGHT}{28cm-#1}
  \setlength{\HEADHEIGHT}{\HEADHEIGHT/2}
  
  \newlength{\MARGIN}
  \setlength{\MARGIN}{16cm-#2}
  \setlength{\MARGIN}{\MARGIN/2}
    
	\topmargin -2.5cm
	\headheight \HEADHEIGHT
	\headsep 0.75cm
	\topskip 0cm 
	\textheight #1
	\textwidth #2
	\oddsidemargin \MARGIN
	\evensidemargin \MARGIN
	\parindent 0.5cm
	\parskip 0.75em plus 0.2em minus 0.1em
}

\marges{26cm}{18cm}

%**************************************************************************************************
% *
% * Début du document
% *
%**************************************************************************************************

\begin{document}

Voici un automate que l'on veut déterminiser :

\begin{center}
  \begin{tikzpicture}
\node[state, initial] (q0) {0};
\node[state, right of=q0] (q1) {1};
\node[state, right of=q1] (q2) {2};
\node[state, accepting, right of=q2] (q3) {3};
\node[state, below of=q0] (q4) {4};
\node[state, right of=q4] (q5) {5};
\node[state, accepting, below of=q3] (q6) {6};

\draw
(q0) edge[above] node{a} (q1)
(q0) edge[left] node{b} (q4)
(q1) edge[above] node{b} (q2)
(q1) edge[right] node{a} (q5)
(q2) edge[above, bend right] node{b} (q0)
(q2) edge[above] node{a} (q3)
(q3) edge[right, bend left] node{a} (q6)
(q3) edge[loop right] node{b} (q3)
(q4) edge[above] node{a} (q5)
(q4) edge[loop below] node{b} (q4)
(q5) edge[above] node{a} (q6)
(q5) edge[loop below] node{b} (q5)
(q6) edge[loop right] node{a} (q6)
(q6) edge[right, bend left] node{b} (q3);
  \end{tikzpicture}
\end{center}

En appliquant l'algorithme de minimisation : 

\begin{center}
\begin{tabular}{|c|c|c|c|c|c|c|c|} \hline
                            & 0 & 1 & 2 & 3 & 4 & 5 & 6 \\ \hline
$\pi'$                      & A & A & A & B & A & A & B \\
$\stackrel{a}{\rightarrow}$ & A & A & B & B & A & B & B \\
$\stackrel{b}{\rightarrow}$ & A & A & A & B & A & A & B \\ \hline
$\pi'$                      & A & A & B & C & A & B & C \\
$\stackrel{a}{\rightarrow}$ & A & B & C & C & B & C & C \\
$\stackrel{b}{\rightarrow}$ & A & B & A & C & A & B & C \\ \hline
$\pi'$                      & A & B & C & D & E & F & D \\
$\stackrel{a}{\rightarrow}$ & B & F & D & D & F & D & D \\
$\stackrel{b}{\rightarrow}$ & E & C & A & D & E & F & D \\ \hline
$\pi'$                      & A & B & C & D & E & F & D \\ \hline
\end{tabular}
\end{center}

Voici l'automate obtenu :

\begin{center}
  \begin{tikzpicture}
\node[state, initial] (A) {A};
\node[state, right of=A] (B) {B};
\node[state, right of=B] (C) {C};
\node[state, accepting, right of=C] (D) {D};
\node[state, below of=A] (E) {E};
\node[state, right of=E] (F) {F};

\draw
(A) edge[above] node{a} (B)
(A) edge[left] node{b} (E)
(B) edge[above] node{b} (C)
(B) edge[right] node{a} (F)
(C) edge[above, bend right] node{b} (A)
(C) edge[above] node{a} (D)
(D) edge[loop right] node{b} (D)
(D) edge[loop above] node{a} (D)
(E) edge[loop below] node{b} (D)
(E) edge[above] node{a} (F)
(F) edge[loop below] node{b} (F)
(F) edge[above] node{a} (D);
  \end{tikzpicture}
\end{center}

\end{document}