\documentclass[12pt,a4paper]{article}
\usepackage[utf8]{inputenc}    % Ces deux packages sont utilisés pour l'encodage des caractères
\usepackage[T1]{fontenc}       % 
\usepackage[french]{babel}     % Ce package permet d'utiliser les environnements en Français

\usepackage{listings}

\usepackage{tikz}
\usetikzlibrary{automata, positioning, arrows}

\tikzset{
  ->, % makes the edges directed
  >=stealth', % makes the arrow heads bold
  node distance=3cm, % specifies the minimum distance between two nodes. Change if necessary.
  every state/.style={thick, fill=gray!10}, % sets the properties for each 'state' node
  initial text=$ $, % sets the text that appears on the start arrow
}

%**************************************************************************************************
% *
% * Configuration des marges
% *
%**************************************************************************************************

\usepackage{calc}
\newcommand{\marges}[2]{
  \newlength{\HEADHEIGHT}
  \setlength{\HEADHEIGHT}{28cm-#1}
  \setlength{\HEADHEIGHT}{\HEADHEIGHT/2}
  
  \newlength{\MARGIN}
  \setlength{\MARGIN}{16cm-#2}
  \setlength{\MARGIN}{\MARGIN/2}
    
	\topmargin -2.5cm
	\headheight \HEADHEIGHT
	\headsep 0.75cm
	\topskip 0cm 
	\textheight #1
	\textwidth #2
	\oddsidemargin \MARGIN
	\evensidemargin \MARGIN
	\parindent 0.5cm
	\parskip 0.75em plus 0.2em minus 0.1em
}

\marges{26cm}{18cm}

%**************************************************************************************************
% *
% * Début du document
% *
%**************************************************************************************************

\begin{document}

\noindent Le fichier \texttt{AFD\_1.aut} :
\begin{lstlisting}[mathescape,frame=single]
AFD
Q={q0,q1,q2,q3}
A={a,b}
s=q0
F={q2,q3}
d={(q0,a,q1)(q0,b,q2)(q1,a,q1)(q1,b,q3)(q2,b,q2)(q2,a,q3)}
\end{lstlisting}
Ce qui donne l'automate suivant :
\begin{figure}[ht]
  \centering
  \begin{tikzpicture}
\node[state, initial] at(0,1) (q0) {$q_0$};
\node[state] at(2,2) (q1) {$q_1$};
\node[state, accepting] at(2,0) (q2) {$q_2$};
\node[state, accepting] at(4,1) (q3) {$q_3$};
\draw 
(q0) edge[bend left, above] node{a} (q1)
(q0) edge[bend right, above] node{b} (q2)
(q1) edge[loop above] node{a} (q1)
(q2) edge[loop below] node{b} (q2)
(q1) edge[bend left, above] node{b} (q3)
(q2) edge[bend right, above] node{a} (q3);
  \end{tikzpicture}
  \caption{Fichier \texttt{AFD\_1.aut}}
\end{figure}

\noindent Le fichier \texttt{AFN\_1.aut} :
\begin{lstlisting}[mathescape,frame=single]
AFN
Q={q0,q1,q2,q3}
s=q0
F={q3}
A={a,b}
D={(q0,$\epsilon$,q1)(q0,$\epsilon$,q2)(q1,a,q1)(q1,b,q3)(q2,b,q2)(q2,b,q3)}
\end{lstlisting}
Ce qui donne l'automate suivant :
\begin{figure}[ht]
  \centering
  \begin{tikzpicture}
\node[state, initial] at(0,1) (q0) {$q_0$};
\node[state] at(2,2) (q1) {$q_1$};
\node[state] at(2,0) (q2) {$q_2$};
\node[state, accepting] at(4,1) (q3) {$q_3$};
\draw 
(q0) edge[bend left, above] node{$\epsilon$} (q1)
(q0) edge[bend right, above] node{$\epsilon$} (q2)
(q1) edge[loop above] node{a} (q1)
(q2) edge[loop below] node{b} (q2)
(q1) edge[bend left, above] node{b} (q3)
(q2) edge[bend right, above] node{b} (q3);
  \end{tikzpicture}
  \caption{Fichier \texttt{AFN\_1.aut}}
\end{figure}

\noindent Le fichier \texttt{AFNP\_1.aut} :
\begin{lstlisting}[mathescape,frame=single]
AFNP
Q={q0,q1}
s=q0
A={a,b}
B={a}
z=$\epsilon$
D={
  (q0,a,$\epsilon$),(q0,a)
  (q0,a,a),(q0,aa)
  (q0,b,a),(q1,$\epsilon$)
  (q1,b,a),(q1,$\epsilon$)
}
\end{lstlisting}

Dans le cours, nous n'avons pas vu la représentation sous forme de graphe d'un automate à pile.

\end{document}